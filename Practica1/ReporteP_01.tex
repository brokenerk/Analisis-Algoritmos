\documentclass[12pt]{article}
\usepackage[utf8]{inputenc}
\usepackage[spanish]{babel}
\decimalpoint
\usepackage{amsmath}
\usepackage{amsthm}
\usepackage{amssymb}
\usepackage{graphicx}
\usepackage[margin=0.9in]{geometry}
\usepackage{fancyhdr}
\usepackage[inline]{enumitem}
\usepackage{float}
\usepackage{cancel}
\usepackage{bigints}
\usepackage{color}
\usepackage{xcolor}
\usepackage{listingsutf8}
\usepackage{algorithm}
\usepackage{tocloft}
\usepackage[none]{hyphenat}
\usepackage{graphicx}
\usepackage{grffile}
\usepackage{tabularx}
\usepackage[nottoc,notlot,notlof]{tocbibind}
\usepackage{times}
\usepackage{color}
\definecolor{gray97}{gray}{.97}
\definecolor{gray75}{gray}{.75}
\definecolor{gray45}{gray}{.45}
\renewcommand{\cftsecleader}{\cftdotfill{\cftdotsep}}
\pagestyle{fancy}
\setlength{\headheight}{15pt} 
\lhead{PRACTICA 1}
\rhead{\thepage}
\lfoot{ESCOM-IPN}
\renewcommand{\footrulewidth}{0.5pt}
\setlength{\parskip}{0.5em}
\newcommand{\ve}[1]{\overrightarrow{#1}}
\newcommand{\abs}[1]{\left\lvert #1 \right\lvert}
\date{26 de febrero de 2017}
\title{Pruebas a posteriori}
\author{Reporte 1}

\definecolor{pblue}{rgb}{0.13,0.13,1}
\definecolor{pgreen}{rgb}{0,0.5,0}
\definecolor{pred}{rgb}{0.9,0,0}
\definecolor{pgrey}{rgb}{0.46,0.45,0.48}
\lstset{tabsize=1}

\usepackage{listings}
\lstset{ frame=Ltb,
framerule=0pt,
aboveskip=0.5cm,
framextopmargin=3pt,
framexbottommargin=3pt,
framexleftmargin=0.4cm,
framesep=0pt,
rulesep=.4pt,
backgroundcolor=\color{gray97},
rulesepcolor=\color{black},
%
stringstyle=\ttfamily,
showstringspaces = false,
basicstyle=\small\ttfamily,
commentstyle=\color{gray45},
keywordstyle=\bfseries,
%
numbers=left,
numbersep=15pt,
numberstyle=\tiny,
numberfirstline = false,
breaklines=true,
}

% minimizar fragmentado de listados
\lstnewenvironment{listing}[1][]
{\lstset{#1}\pagebreak[0]}{\pagebreak[0]}

\lstdefinestyle{consola}
{basicstyle=\scriptsize\bf\ttfamily,
backgroundcolor=\color{gray75},
}

\lstdefinestyle{Java}
{language=Java,
}

%%%%%%%%%%%%%%%%%%%%%

\lstdefinestyle{customc}{
  belowcaptionskip=1\baselineskip,
  breaklines=true,
  frame=L,
  xleftmargin=\parindent,
  language=C,
  showstringspaces=false,
  basicstyle=\footnotesize\ttfamily,
  keywordstyle=\bfseries\color{green!40!black},
  commentstyle=\itshape\color{purple!40!black},
  identifierstyle=\color{blue},
  stringstyle=\color{orange},
}

\lstdefinestyle{customasm}{
  belowcaptionskip=1\baselineskip,
  frame=L,
  xleftmargin=\parindent,
  language=[x86masm]Assembler,
  basicstyle=\footnotesize\ttfamily,
  commentstyle=\itshape\color{purple!40!black},
}

\lstset{escapechar=@,style=customc}


    % =====  CODE EDITOR =========
    \lstdefinestyle{CompilandoStyle} {                              %This is Code Style
        backgroundcolor=\color{BlueGrey800MD},                      %Background Color  
        basicstyle=\tiny\color{white},                              %Font color
        commentstyle=\color{BlueGrey100MD},                         %Comment color
        stringstyle=\color{TealMD},                                 %String color
        keywordstyle=\color{Green100MD},                            %keywords color
        numberstyle=\tiny\color{TealMD},                            %Size of a number
        frame=shadowbox,                                            %Adds a frame around the code
        breakatwhitespace=true,                                     %Style                       
        breaklines=true,                                            %Style                   
        keepspaces=true,                                            %Style                   
        numbers=left,                                               %Style                   
        numbersep=10pt,                                             %Style 
        xleftmargin=\parindent,                                     %Style 
        tabsize=4                                                   %Style 
    }
 
    \lstset{style=CompilandoStyle}                                  %Use this style

    \usepackage{minted} % Paquete que permite citar codigo
    \usemintedstyle{borland} % Aqui se define el colorscheme para minted
    \setminted{
        fontsize = \scriptsize, % Ajusta el codigo a la hoja
        baselinestretch = 1,
        linenos, % set numbers
        breaklines=true, % Hace un salto de linea automatico en caso de que se llege al final de la line
        tabsize=3 
    }

%Permite crear columnas en el documento
\usepackage{multicol} 
\usepackage{color}
\usepackage{comment}
\newcommand{\tabitem}{~~\llap{\textbullet}~~}
\newcommand{\subtabitem}{~~~~\llap{\textbullet}~~}

\bibliographystyle{IEEEtran}
\begin{document}
		\begin{titlepage}
			\begin{center}
				
				% Upper part of the page. The '~' is needed because \\
				% only works if a paragraph has started.
				
				\noindent
				\begin{minipage}{0.5\textwidth}
					\begin{flushleft} \large
						\includegraphics[width=0.3\textwidth]{../ipn.png}
					\end{flushleft}
				\end{minipage}%
				\begin{minipage}{0.55\textwidth}
					\begin{flushright} \large
						\includegraphics[width=0.7\textwidth]{../escom.png}
					\end{flushright}
				\end{minipage}
				
				\textsc{\LARGE Instituto Politécnico Nacional}\\[0.5cm]
				
				\textsc{\Large Escuela Superior de Cómputo}\\[1cm]
				
				% Title
				
				{ \huge Práctica 1 - Pruebas a posteriori\\[1cm] }
				
				{ \Large Unidad de aprendizaje: Análisis de Algoritmos} \\[1cm]
				
				{ \Large Grupo: 3CM3} \\[1cm]
				
				\noindent
				\begin{minipage}{0.5\textwidth}
					\begin{flushleft} \large
						\emph{Alumnos(a):}\\
						
						\begin{tabular}{ll}
					     Nicolás Sayago Abigail\\
					     Ramos Díaz Enrique \\
					\end{tabular}
					\end{flushleft}
				\end{minipage}%
				\begin{minipage}{0.5\textwidth}
					\begin{flushright} \large
						\emph{Profesor(a): Edgardo Adrián Franco Martínez} \\
						  \\
					\end{flushright}
				\end{minipage}
				
				\vfill
				
				% Bottom of the page
				{\large 12 de Septiembre 2018}
			\end{center}
		\end{titlepage}
	
	\tableofcontents
	\newpage
	\section{Planteamiento del problema}


	% /////////////////////////////////////////////////////////
	%			ACTIVIDADES Y PRUEBAS
	% ////////////////////////////////////////////////////////
	
	\section{Actividades y Pruebas}
	
		\subsection{Burbuja simple}

		\subsection{Burbuja Optimizada}

		\subsection{Ordenamiento por inserción}

		\subsection{Ordenamiento por selección}

		\subsection{Ordenamiento Shell}

		\subsection{Árbol binario de búsqueda}
		
	% /////////////////////////////////////////////////////////
	%							ANEXO
	% ////////////////////////////////////////////////////////
	
	\section{Anexo}

		\subsection{Burbuja simple}
		\inputminted{c++}{1.c}

		\subsection{Burbuja Optimizada}
			
		\subsection{Ordenamiento por inserción}
			
		\subsection{Ordenamiento por selección}
			
		\subsection{Ordenamiento Shell}
			
		\subsection{Árbol binario de búsqueda}
			
	% /////////////////////////////////////////////////////////
	%						BIBLIOGRAFIA
	% ////////////////////////////////////////////////////////
	
	\section{Bibliografía}

	\nocite{ref2}
	\bibliography{referencias}
     
\end{document}