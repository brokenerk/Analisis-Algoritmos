\documentclass[12pt]{article}
\usepackage[utf8]{inputenc}
\usepackage[spanish]{babel}
\decimalpoint
\usepackage{amsmath}
\usepackage{caption}
\usepackage{amsthm}
\usepackage{amssymb}
\usepackage{graphicx}
\usepackage[margin=0.9in]{geometry}
\usepackage{fancyhdr}
\usepackage[inline]{enumitem}
\usepackage{float}
\usepackage{cancel}
\usepackage{bigints}
\usepackage{color}
\usepackage{xcolor}
\usepackage{listingsutf8}
\usepackage{algorithm}
\usepackage{tocloft}
\usepackage[none]{hyphenat}
\usepackage{graphicx}
\usepackage{grffile}
\usepackage{tabularx}
\usepackage[nottoc,notlot,notlof]{tocbibind}
\usepackage{times}
\usepackage{color}
\definecolor{gray97}{gray}{.97}
\definecolor{gray75}{gray}{.75}
\definecolor{gray45}{gray}{.45}
\renewcommand{\cftsecleader}{\cftdotfill{\cftdotsep}}
\pagestyle{fancy}
\setlength{\headheight}{15pt} 
\lhead{Práctica 2 - Análisis temporal y notación de orden (Algoritmos de búsqueda)}
\rhead{\thepage}
\lfoot{ESCOM-IPN}
\renewcommand{\footrulewidth}{0.5pt}
\setlength{\parskip}{0.5em}
\newcommand{\ve}[1]{\overrightarrow{#1}}
\newcommand{\abs}[1]{\left\lvert #1 \right\lvert}
\date{26 de febrero de 2017}
\title{Pruebas a posteriori}
\author{Reporte 1}

\definecolor{pblue}{rgb}{0.13,0.13,1}
\definecolor{pgreen}{rgb}{0,0.5,0}
\definecolor{pred}{rgb}{0.9,0,0}
\definecolor{pgrey}{rgb}{0.46,0.45,0.48}
\lstset{tabsize=1}

\usepackage{listings}
\lstset{ frame=Ltb,
framerule=0pt,
aboveskip=0.5cm,
framextopmargin=3pt,
framexbottommargin=3pt,
framexleftmargin=0.4cm,
framesep=0pt,
rulesep=.4pt,
backgroundcolor=\color{gray97},
rulesepcolor=\color{black},
%
stringstyle=\ttfamily,
showstringspaces = false,
basicstyle=\small\ttfamily,
commentstyle=\color{gray45},
keywordstyle=\bfseries,
%
numbers=left,
numbersep=15pt,
numberstyle=\tiny,
numberfirstline = false,
breaklines=true,
}

% minimizar fragmentado de listados
\lstnewenvironment{listing}[1][]
{\lstset{#1}\pagebreak[0]}{\pagebreak[0]}

\lstdefinestyle{consola}
{basicstyle=\scriptsize\bf\ttfamily,
backgroundcolor=\color{gray75},
}

\lstdefinestyle{Java}
{language=Java,
}

%%%%%%%%%%%%%%%%%%%%%

\lstdefinestyle{customc}{
  belowcaptionskip=1\baselineskip,
  breaklines=true,
  frame=L,
  xleftmargin=\parindent,
  language=C,
  showstringspaces=false,
  basicstyle=\footnotesize\ttfamily,
  keywordstyle=\bfseries\color{green!40!black},
  commentstyle=\itshape\color{purple!40!black},
  identifierstyle=\color{blue},
  stringstyle=\color{orange},
}

\lstdefinestyle{customasm}{
  belowcaptionskip=1\baselineskip,
  frame=L,
  xleftmargin=\parindent,
  language=[x86masm]Assembler,
  basicstyle=\footnotesize\ttfamily,
  commentstyle=\itshape\color{purple!40!black},
}

\lstset{escapechar=@,style=customc}


    % =====  CODE EDITOR =========
    \lstdefinestyle{CompilandoStyle} {                              %This is Code Style
        backgroundcolor=\color{BlueGrey800MD},                      %Background Color  
        basicstyle=\tiny\color{white},                              %Font color
        commentstyle=\color{BlueGrey100MD},                         %Comment color
        stringstyle=\color{TealMD},                                 %String color
        keywordstyle=\color{Green100MD},                            %keywords color
        numberstyle=\tiny\color{TealMD},                            %Size of a number
        frame=shadowbox,                                            %Adds a frame around the code
        breakatwhitespace=true,                                     %Style                       
        breaklines=true,                                            %Style                   
        keepspaces=true,                                            %Style                   
        numbers=left,                                               %Style                   
        numbersep=10pt,                                             %Style 
        xleftmargin=\parindent,                                     %Style 
        tabsize=4                                                   %Style 
    }
 
    \lstset{style=CompilandoStyle}                                  %Use this style

    \usepackage{minted} % Paquete que permite citar codigo
    \usemintedstyle{borland} % Aqui se define el colorscheme para minted
    \setminted{
        fontsize = \scriptsize, % Ajusta el codigo a la hoja
        baselinestretch = 1,
        linenos, % set numbers
        breaklines=true, % Hace un salto de linea automatico en caso de que se llege al final de la line
        tabsize=3 
    }

%Permite crear columnas en el documento
\usepackage{multicol} 
\usepackage{color}
\usepackage{comment}
\newcommand{\tabitem}{~~\llap{\textbullet}~~}
\newcommand{\subtabitem}{~~~~\llap{\textbullet}~~}

% ---------------------------------------------------
% 						FONT 
% ---------------------------------------------------

\usepackage{cmbright}								% Font


\begin{document}

% ###########################################################################################
% ----------------------------------- FANCY TITLE PAGE --------------------------------------
% ###########################################################################################
\begin{titlepage}
			\begin{center}
				\noindent
				\begin{minipage}{0.5\textwidth}
					\begin{flushleft} \large
						\includegraphics[width=0.3\textwidth]{../ipn.png}
					\end{flushleft}
				\end{minipage}%
				\begin{minipage}{0.55\textwidth}
					\begin{flushright} \large
						\includegraphics[width=0.5\textwidth]{../escom.png}
					\end{flushright}
				\end{minipage}
				
				\textsc{\LARGE Instituto Politécnico Nacional}\\[0.5cm]
				
				\textsc{\Large Escuela Superior de Cómputo}\\[1cm]
				
				% Title
				
				{ \huge Ejercicio 05 - Análisis de Algoritmos recursivos  \\[1cm] }
				
				{ \Large Unidad de aprendizaje: Análisis de algoritmos} \\[1cm]
				
				{ \Large Grupo: 3CM3 } \\[1cm]
				
				\noindent
				\begin{minipage}{0.5\textwidth}
					\begin{flushleft} \large
						\emph{Alumno(a):}\\
						
						\begin{tabular}{ll}
					     Nicolás Sayago Abigail\\
					\end{tabular}
					\end{flushleft}
				\end{minipage}%
				\begin{minipage}{0.5\textwidth}
					\begin{flushright} \large
						\emph{Profesor(a):} \\
					    Edgardo Adrian Franco  \\
					\end{flushright}
				\end{minipage}
				\vfill
				\begin{minipage}{0.5\textwidth}
					\begin{center} \large
						\includegraphics[width=0.6\textwidth]{Abigail/Images/A.jpg}
					\end{center}
				\end{minipage}
					
				% Bottom of the page
				{\large 10 Septiembre de 2018}
			\end{center}
		\end{titlepage}
	\tableofcontents
  \newpage

  \section{Ejercicio 1}
    Calcula la cota de complejidad para el algoritmos de la siguiente función recursiva.

    % Insertar algoritmo aquí.

  \section{Ejercicio 2}
    Calcular la complejidad de la implementación recursiva del producto.

    % Insertar algoritmo aquí.

  \section{Ejercicio 3}
    Calcular el costo de un recorrido In-orden en un Árbol Binario completamente lleno.

  \section{Ejercicio 4}
    Calcular la cota de complejidad del algoritmo de búsqueda ternaria.


  \section{Ejercicio 5}
    Calcular la cota de complejidad del algoritmo de ordenamiento QuickSort.

  \section{Ejercicio 6}
    Resolver las siguientes ecuaciones y dar su orden de complejidad.

    \begin{itemize}
      \item $T(n) = 3T(n - 1) + 4T(n - 2) \Rightarrow n > 1; T(0) = 0; T(1) = 1 $

        Es una recurrencia homogénea. Reacomodando los términos:
        $$
          T(n) - 3T(n - 1) - 4T(n - 2) = 0
        $$
        
        Sustituyendo por $x$, obtenemos una ecuación a resolver:
        $$
          x^{2} - 3x^{1} -4 = 0
        $$

        Obteniendo las raíces:
        \begin{equation*}
          \begin{split}
            x^{2} - 3x^{1} - 4 & = 0 \\
            (x - 4)(x + 1) & = 0  
          \end{split}
        \end{equation*}

        Las raices son:
        $$ r_{1} = 4 $$
        $$ r_{2} = -1 $$

        La ecuación con recurrencia es:
        \begin{equation*}
          \begin{split}
            C_{1}r_{1}^{n} & + C_{2}r_{2}^{n} \\
            C_{1}4^{n} & + C_{2}(-1)^{n}
          \end{split}
        \end{equation*}

        Para encontrar $C_{1}$ y $C_{2}$, tomamos los casos base $ T(0) = 0; T(1) = 1 $:
        \begin{equation*}
          \begin{split}
            T(0) & = C_{1}4^{0} + C_{2}(-1)^{0} = 0   \\
                 & = C_{1} + C_{2} = 0                \\
            T(1) & = C_{1}4^{1} + C_{2}(-1)^{1} = 1   \\
                 & = 4C_{1} - C_{2} = 1                \\
          \end{split}
        \end{equation*}

        Tenemos el siguiente sistema de ecuaciones:
        \begin{equation*}
          \begin{split}
            C_{1} + C_{2} & = 0                \\
            4C_{1} - C_{2} & = 1 
          \end{split}
        \end{equation*}

        Resolviendo:
        \begin{equation*}
          \begin{split}
            C_{1} & = -C_{2}                 \\
            4(-C_{2}) - C_{2} & = -5C_{2} = 1 
          \end{split}
        \end{equation*}

        Los valores de las constantes son:
        $$C_{1} = \frac{1}{5} $$
        $$C_{2} = -\frac{1}{5} $$
        
        Sustituyendo:
        \begin{equation*}
          \begin{split}
            C_{1}4^{n} & + C_{2}(-1)^{n}  \\
            \frac{1}{5}4^{n} & - \frac{1}{5}(-1)^{n}        
          \end{split}
        \end{equation*}

        Su orden de complejidad es:
        $$ O() $$

      \item $T(n) = T(n - 1) + 4T(n - 2) + (n + 5)2^{n} \Rightarrow n > 1; T(0) = 5; T(1) = 27 $
        
        Es una recurrencia no homogénea. Reordenando los términos:
        \begin{equation*}
          \begin{split}
            T(n) - T(n - 1) - 4T(n - 2) = (n + 5)2^{n}                
          \end{split}
        \end{equation*}

        Obteniendo la ecuación caracteristica:
        \begin{equation*}
          \begin{split}
            \left(x^{2} - x - 4\right) \left( x - 2\right)^{2} 
          \end{split}
        \end{equation*}




        \begin{equation*}
          \begin{split}
          \end{split}
        \end{equation*}


      \item $T(n) - 2T(n - 1) = 3^{n} \Rightarrow n \geq 2; T(0) = 0, T(1) = 1$
    \end{itemize}

  \section{Ejercicio 7}
    Calcular la cota de complejidad que tendrían los algoritmos con los siguientes modelos recurrentes.

    \begin{itemize}
      \item $ T(n) = T(\frac{n}{3}) + 4T(\frac{n}{2}) + 2n^{2} + n $
        
        Usando el Teorema Maestro:

        Dividimos el problema y primero tenemos:
        $$ T(n) = T(\frac{n}{3}) + 2n^{2} $$

        Identificamos que:
        $$ a = 1 $$
        $$ b = 3 $$
        $$ f(n) = 2n^{2} $$ 

        Sustituimos en el caso 1:
        \begin{equation*}
          \begin{split}
            f(n) = O( n^{log_{b}^{a}-\epsilon}) = O( n^{log_{3}^{1}-\epsilon}) = O( n^{0-\epsilon})
          \end{split}
        \end{equation*}

        Observamos que en ese caso no funcionaria con algun $\epsilon > 0$. En el caso 2 no se prueba puesto que se puede ignorar. Probamos el caso 3:
        
        \begin{equation*}
          \begin{split}
            f(n) = \Omega( n^{log_{b}^{a} + \epsilon}) = \Omega( n^{log_{3}^{1} + \epsilon}) = \Omega( n^{0 + \epsilon}) = \Omega( n^{0 + 2}) = \Omega( n^{2}) 
          \end{split}
        \end{equation*}

        Con $\epsilon = 2$ se cumple el caso 3. Por lo tanto:

        $$ \Theta\left( n^{2} \right) $$


        Trabajamos la segunda parte del problema

        $$ T(n) = 4T(\frac{n}{2}) + n $$

        Identificamos que:
        $$ a = 4 $$
        $$ b = 2 $$
        $$ f(n) = n $$ 

        Sustituimos en el caso 1:
        \begin{equation*}
          \begin{split}
            f(n) = O( n^{log_{b}^{a}-\epsilon}) = O( n^{log_{3}^{1}-\epsilon}) = O( n^{0-\epsilon})
          \end{split}
        \end{equation*}

        Observamos que en ese caso no funcionaria con algun $\epsilon > 0$. En el caso 2 no se prueba puesto que se puede ignorar. Probamos el caso 3:
        
        \begin{equation*}
          \begin{split}
            f(n) = \Omega( n^{log_{b}^{a} + \epsilon}) = \Omega( n^{log_{3}^{1} + \epsilon}) = \Omega( n^{0 + \epsilon}) = \Omega( n^{0 + 2}) = \Omega( n^{2}) 
          \end{split}
        \end{equation*}

        Con $\epsilon = 2$ se cumple el caso 3. Por lo tanto:

        $$ \Theta\left( n^{2} \right) $$

      \item $ T(n) = T(n - 1) + T(n - 2) + T(\frac{n}{2})$ si $n>1$;  $T(0) = 1$, $T(1) = 1$

      Por Teorema Maestro:






      \item $ T(n) = T(\frac{n}{2} + 2T(\frac{n}{4}) + 2)$
      \item $ T(n) = 2T(\frac{n}{2}) + 4T(\frac{n}{4}) + 10n^{2} + 5n $
    \end{itemize}

    
\end{document}