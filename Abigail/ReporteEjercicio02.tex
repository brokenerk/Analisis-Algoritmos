
% ###########################################################################################
% --------------------------------- DOCUMENT CONFIGURATION ----------------------------------
% ###########################################################################################

\documentclass[12pt, letterpaper]{article}			% Font size, paper size and type of document
\usepackage[utf8]{inputenc}							% To allow characters beyond ASCII
\usepackage[spanish]{babel}							% Changes the language  and display special characters
\usepackage{comment}								% Allow make comments
\usepackage[encoding]{inputenc}						% To set up input encoding

% ---------------------------------------------------
% 						FONT 
% ---------------------------------------------------

\usepackage{cmbright}								% Font

% ###########################################################################################
% ------------------------------------- FORMATTING ------------------------------------------
% ###########################################################################################

% ---------------------------------------------------
% 						LENGTHS
% ---------------------------------------------------

\setlength{\headheight}{15pt}						% Head 
\setlength{\parskip}{0.5em}							% Sets the paragraph separation 		

% ---------------------------------------------------
% 					HEADERS AND FOOTERS
% ---------------------------------------------------

\usepackage{fancyhdr}								% Fancy headers
\pagestyle{fancy}									% Sets the page style called headings
\rfancyhf{}											% Clears the header and footer
\rhead{ \thepage }									% Right header
\lhead{ Title }										% Left header
\rfoot{ ESCOM IPN }									% Right footer
\renewcommand{\footrulewidth}{0.5pt}				% Decorative line for footer
\renewcommand{\headrulewidth}{0.5pt}				% Decorative line for header

% ---------------------------------------------------
% 					PAGE NUMBERING
% ---------------------------------------------------

\pagenumbering{arabic}								% Numbering styles

% ---------------------------------------------------
% 					   PARAGRAPH
% ---------------------------------------------------

% Line spacing
% \renewcommand{\baselinestretch}{1.5}

% ---------------------------------------------------
% 					   MULTIPLE COLUMNS
% ---------------------------------------------------

\usepackage{multicol}								% Allow multiple columns 
\setlength{\columnsep}{1cm}							% Distance between columns
% \def\columnseprulecolor{\color{blue}}				% Inserting vertical rules

% ###########################################################################################
% --------------------------- CODE DOCUMENTATION CONFIGURATION <3 ---------------------------
% ###########################################################################################

% Make nice code documentation
\usepackage{listings}
% Coloooor
\usepackage{color}
\usepackage{xcolor}



% Put in your document, importing code from a file:
% \lstinputlisting[language=Octave]{BitXorMatrix.m}
% ---------------------------------------------------
% 					GENERAL INFORMATION
% ---------------------------------------------------
\date{ - Date -}
\title{ - Title - }
\author{- Author -}




% ###########################################################################################
% --------------------------------- MATHS CONFIGURATION -------------------------------------
% ###########################################################################################

\usepackage{mathtools}
\usepackage{amsmath}
\usepackage{amsthm}
\usepackage{amssymb}
\usepackage{graphicx}
\usepackage[margin=0.9in]{geometry}

\usepackage[inline]{enumitem}
\usepackage{float}
\usepackage{cancel}
\usepackage{bigints}
\usepackage{listingsutf8}
\usepackage{algorithm}
\usepackage{tocloft}
\usepackage[none]{hyphenat}
\usepackage{graphicx}
\usepackage{grffile}
\usepackage{tabularx}
\usepackage[nottoc,notlot,notlof]{tocbibind}
\usepackage{times}
\usepackage{color}
\decimalpoint
\definecolor{gray97}{gray}{.97}
\definecolor{gray75}{gray}{.75}
\definecolor{gray45}{gray}{.45}
\renewcommand{\cftsecleader}{\cftdotfill{\cftdotsep}}

\newcommand{\ve}[1]{\overrightarrow{#1}}
\newcommand{\abs}[1]{\left\lvert #1 \right\lvert}


\newcommand{\tabitem}{~~\llap{\textbullet}~~}
\newcommand{\subtabitem}{~~~~\llap{\textbullet}~~}

\usepackage{color}
\definecolor{gray97}{gray}{.97}
\definecolor{gray75}{gray}{.75}
\definecolor{gray45}{gray}{.45}
\renewcommand{\cftsecleader}{\cftdotfill{\cftdotsep}}
\pagestyle{fancy}
\setlength{\headheight}{15pt} 
\lhead{Complejidad temporal}
\rhead{\thepage}
\lfoot{ESCOM-IPN}
\renewcommand{\footrulewidth}{0.5pt}
\setlength{\parskip}{0.5em}
\newcommand{\ve}[1]{\overrightarrow{#1}}
\newcommand{\abs}[1]{\left\lvert #1 \right\lvert}
\date{05 de Agosto 2019}
\title{Complejidad temporal}
\author{Ejercicio 02}

\definecolor{pblue}{rgb}{0.13,0.13,1}
\definecolor{pgreen}{rgb}{0,0.5,0}
\definecolor{pred}{rgb}{0.9,0,0}
\definecolor{pgrey}{rgb}{0.46,0.45,0.48}
\lstset{tabsize=1}

\usepackage{listings}
\lstset{ frame=Ltb,
framerule=0pt,
aboveskip=0.5cm,
framextopmargin=3pt,
framexbottommargin=3pt,
framexleftmargin=0.4cm,
framesep=0pt,
rulesep=.4pt,
backgroundcolor=\color{gray97},
rulesepcolor=\color{black},
%
stringstyle=\ttfamily,
showstringspaces = false,
basicstyle=\small\ttfamily,
commentstyle=\color{gray45},
keywordstyle=\bfseries,
%
numbers=left,
numbersep=15pt,
numberstyle=\tiny,
numberfirstline = false,
breaklines=true,
}

% minimizar fragmentado de listados
\lstnewenvironment{listing}[1][]
{\lstset{#1}\pagebreak[0]}{\pagebreak[0]}

\lstdefinestyle{consola}
{basicstyle=\scriptsize\bf\ttfamily,
backgroundcolor=\color{gray75},
}

\lstdefinestyle{Java}
{language=Java,
}

%%%%%%%%%%%%%%%%%%%%%

\lstdefinestyle{customc}{
  belowcaptionskip=1\baselineskip,
  breaklines=true,
  frame=L,
  xleftmargin=\parindent,
  language=C,
  showstringspaces=false,
  basicstyle=\footnotesize\ttfamily,
  keywordstyle=\bfseries\color{green!40!black},
  commentstyle=\itshape\color{purple!40!black},
  identifierstyle=\color{blue},
  stringstyle=\color{orange},
}

\lstdefinestyle{customasm}{
  belowcaptionskip=1\baselineskip,
  frame=L,
  xleftmargin=\parindent,
  language=[x86masm]Assembler,
  basicstyle=\footnotesize\ttfamily,
  commentstyle=\itshape\color{purple!40!black},
}

\lstset{escapechar=@,style=customc}

%Permite crear columnas en el documento
\usepackage{multicol} 
\usepackage{color}
\usepackage{comment}
\newcommand{\tabitem}{~~\llap{\textbullet}~~}
\newcommand{\subtabitem}{~~~~\llap{\textbullet}~~}


\begin{document}

% ###########################################################################################
% ----------------------------------- FANCY TITLE PAGE --------------------------------------
% ###########################################################################################
\begin{titlepage}
			\begin{center}
				\noindent
				\begin{minipage}{0.5\textwidth}
					\begin{flushleft} \large
						\includegraphics[width=0.7\textwidth]{../ipn.png}
					\end{flushleft}
				\end{minipage}%
				\begin{minipage}{0.55\textwidth}
					\begin{flushright} \large
						\includegraphics[width=0.5\textwidth]{../escom.png}
					\end{flushright}
				\end{minipage}
				
				\textsc{\LARGE Instituto Politécnico Nacional}\\[0.5cm]
				
				\textsc{\Large Escuela Superior de Cómputo}\\[1cm]
				
				% Title
				
				{ \huge Ejercicio 02 - Complejidad temporal y análisis de casos  \\[1cm] }
				
				{ \Large Unidad de aprendizaje: Teoría computacional} \\[1cm]
				
				{ \Large Grupo: 3CM3 } \\[1cm]
				
				\noindent
				\begin{minipage}{0.5\textwidth}
					\begin{flushleft} \large
						\emph{Alumno(a):}\\
						
						\begin{tabular}{ll}
					     Nicolás Sayago Abigail\\
					\end{tabular}
					\end{flushleft}
				\end{minipage}%
				\begin{minipage}{0.5\textwidth}
					\begin{flushright} \large
						\emph{Profesor(a):} \\
					    Edgardo Adrian Franco  \\
					\end{flushright}
				\end{minipage}
				
				\vfill
				
				% Bottom of the page
				{\large 05 de Agosto de 2018}
			\end{center}
		\end{titlepage}
	\tableofcontents

% ///////////////////////////////////////////////////////////////////////////
%									SECCIÓN A
% ///////////////////////////////////////////////////////////////////////////
	\section{Sección A}
	En esta sección se determinara la función de complejidad temporal y espacial en términos de n. Se consideran las operaciones de asignación, aritméticas, condicionales y saltos implícitos.

	% ----------------------------------------------------------------------
	% 															EJERCICIO 1
	% ----------------------------------------------------------------------
	    \subsection{Ejercicio 1}
	        \subsubsection{Algoritmo}
	        \begin{lstlisting}[style=Java]

    		\end{lstlisting}

	% ----------------------------------------------------------------------
	% 															EJERCICIO 2
	% ----------------------------------------------------------------------
	    \subsection{Ejercicio 2}
		    \subsubsection{Algoritmo}
	        \begin{lstlisting}[style=Java]

    		\end{lstlisting}

   	% ----------------------------------------------------------------------
	% 															EJERCICIO 3
	% ----------------------------------------------------------------------
	    \subsection{Ejercicio 3}
	    	\subsubsection{Algoritmo}
	        \begin{lstlisting}[style=Java]

    		\end{lstlisting}

   	% ----------------------------------------------------------------------
	% 															EJERCICIO 4
	% ----------------------------------------------------------------------    
	    \subsection{Ejercicio 4}
	    	\subsubsection{Algoritmo}
	        \begin{lstlisting}[style=Java]

    		\end{lstlisting}

    % ----------------------------------------------------------------------
	% 															EJERCICIO 5
	% ----------------------------------------------------------------------
	    \subsection{Ejercicio 5}
	    	\subsubsection{Algoritmo}
	        \begin{lstlisting}[style=Java]

    		\end{lstlisting}


% ///////////////////////////////////////////////////////////////////////////
%									SECCIÓN B
% ///////////////////////////////////////////////////////////////////////////	
	\section{Sección B}
	Para los siguiente 3 algoritmos se determinara el número de veces que imprime la palabra "Algoritmos". 

    % ----------------------------------------------------------------------
	% 															EJERCICIO 6
	% ----------------------------------------------------------------------
	    \subsection{Ejercicio 6}
			\subsubsection{Algoritmo}	    
    			\begin{lstlisting}[style=Java]
    		    
    			\end{lstlisting}
    		
    		\subsection{Análisis}

    		\subsection{Tabla}
	        
	        \subsection{Capturas de comprobación}

	        \subsubsection{Código}
	            \begin{lstlisting}[style=Java]
    		    \end{lstlisting}

	% ----------------------------------------------------------------------
	% 															EJERCICIO 7
	% ----------------------------------------------------------------------
	    \subsection{Ejercicio 7}

			\subsubsection{Algoritmo}	    
    			\begin{lstlisting}[style=Java]
    		    \end{lstlisting}

    		\subsection{Análisis}

    		\subsection{Tabla}
	        
	        \subsection{Capturas de comprobación}

	        \subsubsection{Código}
	            \begin{lstlisting}[style=Java]
    		    \end{lstlisting}


	% ----------------------------------------------------------------------
	% 															EJERCICIO 8
	% ----------------------------------------------------------------------
	    \subsection{Ejercicio 8}

			\subsubsection{Algoritmo}	    
    			\begin{lstlisting}[style=Java]
    		    \end{lstlisting}		
    		
    		\subsection{Análisis}

    		\subsection{Tabla}
	        
	        \subsection{Capturas de comprobación}

	        \subsubsection{Código}
	            \begin{lstlisting}[style=Java]
    		    \end{lstlisting}

% ///////////////////////////////////////////////////////////////////////////
%									SECCIÓN C
% ///////////////////////////////////////////////////////////////////////////	
	\section{Sección C}
	En esta sección se determinan las funciones de compeljidad temporal, para el \textbf{mejor caso}, \textbf{peor caso} y \textbf{caso medio}. Se indican cuales son las condiciones de instancia de entrada del peor caso y cuales la del mejor caso.
	Las operaciones básicas son las comparaciones entre elementos del arreglo y asignaciones.


	% ----------------------------------------------------------------------
	% 															EJERCICIO 9
	% ----------------------------------------------------------------------
    	\subsection{Ejercicio 9}
			\textbf{Operaciones básicas:} Se tomara en cuenta la comparación entre elementos del arreglo [A] así como las asignaciones a mayor 1, mayor 2. 
			\subsubsection{Algoritmo}
			    \begin{lstlisting}[style=Java]
func Producto2Mayores(A, n)
	if (A[1] > A[2])
		mayor1 = A[1];
		mayor2 = A[2];
	else
		mayor1 = A[2];
		mayor2 = A[1];

	i = 3;

	while(i <= n)
		if(A[i] > mayor1)
			mayor2 = mayor1;
			mayor1 = A[i];

		else if (A[i] > mayor2)
			mayor2 = A[i];

		i = i + 1;
	return = mayor1 * mayor2;
		    \end{lstlisting}

    		\subsubsection{Análisis}
	    		\begin{itemize}
	    			\item[\Checklist] Mejor caso \\
	    			El mejor caso sucede cuando los dos números mayores son los primeros en el arreglo.

	    			\item[\Checklist] Peor caso \\
					
					\item[\Checklist] Caso medio \\

				\end{itemize}

	        \subsection{Capturas de comprobación}

	        \subsubsection{Código}
	            \begin{lstlisting}[style=Java]
    		    \end{lstlisting}

	% ----------------------------------------------------------------------
	% 															EJERCICIO 10
	% ----------------------------------------------------------------------
	    \subsection{Ejercicio 10}	 

	    \textbf{Operaciones básicas:}
			\subsubsection{Algoritmo}
			    \begin{lstlisting}[style=Java]
    		    \end{lstlisting}

    		\subsubsection{Análisis}
	    		\begin{itemize}
	    			\item[\Checklist] Mejor caso \\

	    			\item[\Checklist] Peor caso \\
					
					\item[\Checklist] Caso medio \\

				\end{itemize}
	        \subsection{Capturas de comprobación}

	        \subsubsection{Código}
	            \begin{lstlisting}[style=Java]
    		    \end{lstlisting}

	% ----------------------------------------------------------------------
	% 															EJERCICIO 11
	% ----------------------------------------------------------------------
	    \subsection{Ejercicio 11}

		\textbf{Operaciones básicas:}
			\subsubsection{Algoritmo}
			    \begin{lstlisting}[style=Java]
    		    \end{lstlisting}

			\subsubsection{Análisis}
	    		\begin{itemize}
	    			\item[\Checklist] Mejor caso \\

	    			\item[\Checklist] Peor caso \\
					
					\item[\Checklist] Caso medio \\

				\end{itemize}	        
	        \subsection{Capturas de comprobación}

	        \subsubsection{Código}
	            \begin{lstlisting}[style=Java]
    		    \end{lstlisting}

	% ----------------------------------------------------------------------
	% 															EJERCICIO 12
	% ----------------------------------------------------------------------
	    \subsection{Ejercicio 12}
		\textbf{Operaciones básicas:}
		
			\subsubsection{Algoritmo}
			    \begin{lstlisting}[style=Java]
    		    \end{lstlisting}

	        \subsubsection{Análisis}
	    		\begin{itemize}
	    			\item[\Checklist] Mejor caso \\

	    			\item[\Checklist] Peor caso \\
					
					\item[\Checklist] Caso medio \\

				\end{itemize}
	        \subsection{Capturas de comprobación}

	        \subsubsection{Código}
	            \begin{lstlisting}[style=Java]
    		    \end{lstlisting}

	% ----------------------------------------------------------------------
	% 															EJERCICIO 13
	% ----------------------------------------------------------------------
	    \subsection{Ejercicio 13}
		\textbf{Operaciones básicas:}
			\subsubsection{Algoritmo}
			    \begin{lstlisting}[style=Java]
    		    \end{lstlisting}

			\subsubsection{Análisis}
	    		\begin{itemize}
	    			\item[\Checklist] Mejor caso \\

	    			\item[\Checklist] Peor caso \\
					
					\item[\Checklist] Caso medio \\

				\end{itemize}	        
	        \subsection{Capturas de comprobación}

	        \subsubsection{Código}
	            \begin{lstlisting}[style=Java]
    		    \end{lstlisting}

	% ----------------------------------------------------------------------
	% 															EJERCICIO 14
	% ----------------------------------------------------------------------
	    \subsection{Ejercicio 14}
		\textbf{Operaciones básicas:}
			\subsubsection{Algoritmo}
			    \begin{lstlisting}[style=Java]
    		    \end{lstlisting}

    		\subsubsection{Análisis}
	    		\begin{itemize}
	    			\item[\Checklist] Mejor caso \\

	    			\item[\Checklist] Peor caso \\
					
					\item[\Checklist] Caso medio \\

				\end{itemize}

	        \subsection{Capturas de comprobación}

	        \subsubsection{Código}
	            \begin{lstlisting}[style=Java]
    		    \end{lstlisting}

	% ----------------------------------------------------------------------
	% 															EJERCICIO 15
	% ----------------------------------------------------------------------
	    \subsection{Ejercicio 15}
	   	\textbf{Operaciones básicas:}
		
			\subsubsection{Algoritmo}
			    \begin{lstlisting}[style=Java]
    		    \end{lstlisting}

    		\subsubsection{Análisis}
	    		\begin{itemize}
	    			\item[\Checklist] Mejor caso \\

	    			\item[\Checklist] Peor caso \\
					
					\item[\Checklist] Caso medio \\

				\end{itemize}

	        \subsection{Capturas de comprobación}

	        \subsubsection{Código}
	            \begin{lstlisting}[style=Java]
    		    \end{lstlisting}

\end{document}